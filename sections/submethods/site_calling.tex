\documentclass[../../main.tex]{subfiles}

\begin{document}
To improve algorithmic efficiency we wish only to consider sites with a non-trivial posterior probability of containing a somatic mutation. Furthermore it has been shown that combining low coverage sequencing data across samples at a locus can decrease false positive rates~\cite{ledurbin}. We therefore must reject loci where the posterior probability of mutation is low. For a given locus $i$:
\begin{equation}
P(SNV_i\mid D_i) = 1- P\left(\sum_{j=1}^m g_{ij} = 0 \mid D_i\right) = 1-P(\sigma = 0 \mid D_i)
\end{equation}
Using Bayes' formula:
\begin{equation}\label{eq:sitebayes}
P\left(\sigma \mid D_i\right) = \frac{P(D_i\mid \sigma)P(\sigma)}{\sum_{\sigma'=0}^{2m}[P(D_i\mid \sigma=\sigma')P(\sigma')]}
\end{equation}
The value of $P(D_i\mid \sigma = 0)$ is simply the product of the cell likelihoods of homozygous reference calculated above. The priors $P(\sigma)$ are those determined by Equation~\eqref{eq:overallprior}. To compute the denominator, howevever, we must compute the likelihood for each alternate allele count across a locus. There are various permutations of cell genotypes that may give rise to an alternate allele count of $\sigma$, so this is not as simple as the special case where $\sigma=0$.\\

Let the phased genotypes of all $m$ cells at a site be represented by $\vec{G} = (G_1,G_2,\dots,G_m)$ where $G_j\in [0,1]\times[0,1]$ is the phased genotype for cell $j$ (0 = reference, 1 = alternate). Furthermore let the unphased genotype vector be $\vec{g} = (g_1,g_2,\dots,g_m)$ be such that $g_j = \norm{G_j}_1$. Our likelihood for $\sigma$ can therefore be considered
\begin{equation} \label{eq:phasedlike}
P(D_i\mid\sigma_i) = \sum_{\vec{G}} P(D_i \mid \vec{G}) P(\vec{G} \mid \sigma_i)
\end{equation}
We assume that all phased genotype vectors with a total alternate allele count of $\sigma$ are equally probable. Since there are $\binom{2m}{\sigma}$ different phased genotype vectors with total alternate allele count $\sigma$, then for any such $\vec{G}$:
\begin{equation*}
P(\vec{G}\mid \sigma) = \binom{2m}{\sigma}^{-1}
\end{equation*}
Since we do not consider phased sequencing data, we must reproduce Equation~\eqref{eq:phasedlike} in an unphased form. To begin, we see that the likelihood $P(D_i\mid \vec{G}) = P(D_i\mid \vec{g})$ if $\vec{g}$ is the unphased vector that corresponds to $\vec{G}$, since our cell genotype likelihoods do not consider phasing. Note that there are $2^\chi$ phased genotype vectors that correspond to any given unphased genotype vector $\vec{g}$, where $\chi(\vec{g})$ is the number of heterozygous cells in the vector. Using this multiplicity, we can now reproduce Equation~\eqref{eq:phasedlike} without reference to phasing.
\begin{equation*}
P(D_i\mid\sigma_i) = \sum_{\vec{g}} \frac{2^{\chi(\vec{g})}}{\binom{2m}{\sigma_i}} P(D_i\mid\vec{g}) = \sum_{\vec{g}} \frac{2^{\chi(\vec{g})}}{\binom{2m}{\sigma_i}} \prod_{j=1}^{m}P(D_{ij}\mid g_{j})
\end{equation*}
Let the function $\delta(\vec{g},\sigma) = 1$ if $\norm{\vec{g}}=\sigma$ otherwise it evaluates to 0. We can now write the above in a more suggestive form:
\begin{equation}\label{eq:sitelikelihood}
P(D_i\mid\sigma_i) = \binom{2m}{\sigma_i}^{-1}\sum_{g_1=0}^2\sum_{g_2=0}^2\dots\sum_{g_m=0}^2 \delta((g_1,\dots g_m),\sigma_i)\left[\prod_{j=1}^{m}\binom{2}{g_j}P(D_{ij}\mid g_{j})\right]
\end{equation}
As has been done previously, we can employ a dynamic programming approach to compute these likelihoods for $\sigma$ from cell genotype likelihoods~\cite{monovar, sciphi, ledurbin}. If we let $F(k,l)$ be the subproblem objective given by
\begin{equation}
F(k,l) = \begin{cases} \sum_{g_1=0}^2\sum_{g_2=0}^2\dots\sum_{g_k=0}^2 \delta((g_1,\dots g_k),l)\left[\prod_{j=1}^{k}\binom{2}{g_j}P(D_{ij}\mid g_{j})\right] \quad 0\leq l \leq 2k \\
0 \qquad\qquad\qquad\qquad\qquad\qquad\qquad\qquad\qquad\qquad\qquad\qquad\qquad\qquad \text{else} \end{cases}
\end{equation}
We can consider creating a genotype vector of length $k$ from a vector of length $k-1$ by adding one new cell with an alternate allele count of $0,1$ or $2$. Hence our recurrence relation can be given by
\begin{equation}
F(k,l) = F(k-1,l)P(D_{ik}\mid g_k = 0) + 2F(k-1,l-1)P(D_{ik}\mid g_k = 1) + F(k-1,l-2)P(D_{ik}\mid g_k = 2)
\end{equation}
Note that two possible phased genotypes correspond to the heterozygous case, hence the factor of 2 in the second term. The base case where $k=1$ corresponds to a single cell
\begin{equation*}
F(1,0) = P(D_{i1}\mid g_1 = 0),\;\; F(1,1) = 2P(D_{i1}\mid g_1=1),\;\; F(1,2) = P(D_{i1}\mid g_1 = 2)
\end{equation*}
The values for $F(k,l)$ are memoized in an array and the likelihood given in Equation~\ref{eq:sitelikelihood} can be given by
\begin{equation}
P(D_i\mid \sigma_i)=\frac{F(m,\sigma_i)}{\binom{2m}{\sigma_i}}
\end{equation}
In this way we can determine the likelihood of all $0\leq \sigma\leq 2m$ which when the priors $P(\sigma)$ compose the sum in Equation~\eqref{eq:sitebayes}.

Sites which have a posterior probability of being variant greater than 0.5(???) will be called as variant candidates.
%TODO important parameter! how many to call?? will have to test this... Call fewer and faster tree joining, call more and more data to work with... also possibly more noise.
%\subsection{Summarisation of ignored loci}
%don't know if this is needed, apart from perhaps a simple count. These will all be considered homo ref, right?
\end{document}
