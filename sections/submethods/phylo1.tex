\documentclass[../../main.tex]{subfiles}

\begin{document}

From the matrix of pairwise evolutionary distances $d$ we can now build a single phylogeny to relate the $m$ cells.
We will then use this phylogeny to improve our SNV calls.

Neighbor joining is an agglomerative phylogeny building algorithm that assumes approximately additive evolutionary pairwise distances.
The edge lengths of a tree are additive if and only if "the distance between any pair of leaves is the sum of the lengths of the edges on the path connecting them"~\cite{BSA}.
Neighbor joining finds pairs of neighboring nodes (initially single cells), connects them and then replaces the two with a new node representing the newly formed cluster.
The distance between the new node and all remaining nodes can be computed using the assumption of additivity.

Note that neighboring nodes are not necessarily those with the closest distance between them as this fails to join neighbors where one edge is short and the other is long~\cite{NJ, BSA}.
Instead, a pair of nodes is selected to be joined in any iteration if they minimize the pairwise metric:
\begin{equation}\label{eq:NJD}
    D_{ij} = d_{ij} - r_i - r_j
\end{equation}
where
\begin{equation*}
    r_i = \frac{1}{|L|-2}\sum_{k\in L}d_{ik}
\end{equation*}
and $L$ is the set of leaves or active nodes to be joined.

Since our approximate distances in reality are not always additive, it is possible to get negative edge lengths.
In this case we set the negative distance to 0, and adjust the adjacent distance so that their sum remains unchanged~\cite{negedge}.
The neighbor joining algorithm is therefore as described below, using similar notation as is used in~\cite{BSA}.

\begin{algorithm}
    \caption{Neighbor Joining. Modified from Saitou \& Nei, 1987}
    \begin{algorithmic}[5]
        \State $L\gets$ \{Cells, reference outgroup\}
        \State $T\gets\{\}$
        \While{$|L| > 2$}
            \State $D\gets$ Eq. \eqref{eq:NJD} on all pairs in $L$
            \State $N_i,N_j\gets$ nodes from $L$ s.t. $(i,j) = \text{argmin } D_{ij}$
            \State $N_k \gets$ new node
            \ForAll{$N_l \in L - \{N_i, N_j\}$}
                \State $d_{kl} \gets \frac{1}{2}\left(d_{il} + d_{jl} - d_{ij}\right)$
            \EndFor
            \State $T \gets T + N_k$ with edge distances $d_{ik} = \frac{1}{2}\left(d_{ij}+r_i-r_j\right)$ and $d_{jk} = \frac{1}{2}\left(d_{ij}+r_j-r_i\right)$
            \If{$d_{ik} < 0$  \textbf{or} $d_{jk} <0$}
                \State Positive $d_+\gets d_+ + d_-$ 
                \State Negative $d_-\gets 0$ 
            \EndIf
            \State $L \gets L -N_i -N_j + N_k$
        \EndWhile
        \State Join last two nodes with edge distance $d_{ij}$
    \end{algorithmic}
\end{algorithm}

After the neighbor joining algorithm is complete, the tree can be rooted using the pseudo-cell that is homozygous reference with a probability of 1.



\end{document}
