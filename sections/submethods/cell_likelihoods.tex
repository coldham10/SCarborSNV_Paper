\documentclass[../../main.tex]{subfiles}

\begin{document}
%TODO XXX priors before this not finally checked for writing

Having found the prior probability distribution for the alternate allele count across all cells at a locus, we now turn to the likelihoods of individual cell genotypes.
The (unphased) genotype $g\in\{0,1,2\}$ for cell $j$ at locus $i$ is equivalent to the specific alternate allele count for that cell.
At this stage of the calculation we consider all sites and cells independently.
\subsubsection*{Homozygous genotypes}
For the cell likelihood $\Lambda(g) = P(d_{ij}\,\mid\,g)$ we assume the error on each read to be independent of the other reads.
Hence if cell $j$ has $l$ reads covering locus $i$ we have:
\begin{equation*}
    P(d_{ij}\mid g) = \prod_{k=1}^l P(d_{ijk} \mid g)
\end{equation*}
Note $d_{ij}=(\vec{r},\vec{e})$, where $\vec{r}$ are the $l$ reads for this cell at this locus and $\vec{e}$ are the associated $l$ probabilities of read error.
The values of $\vec{e}$ are provided by the sequencing machine in the form of Phred quality scores, which are directly proportional to the log-probability of sequencing error. 

For any particular read and associated read error probability, we first marginalize the probability of observing this data on whether or not there has been an error in sequencing:
\begin{equation*}
    P(d_{k} \mid g) =  P(r_k,\,se \mid g) + P(r_k,\,\neg se \mid g)
\end{equation*}
Since errors can occur during amplification or sequencing, we model the result of amplification as a latent ``intermediate allele", denoted $\beta$.
This random variable follows a discrete, symmetric, four-point distribution that assumes a constant probability of amplification error for any base, and equal probability of amplifying either allele in the heterozygous case~\cite{monovar}.
The probability of amplification error is an input parameter for SCarborSNV, with a default value of 0.002.
If a read is not the result of a sequencing error:
\begin{equation*}
    P(r_k,\,\neg se \mid g) = P(r_k\mid \neg se ,\, g)(1-e_k)=P(\beta_k=r_k\mid g)(1-e_k) 
\end{equation*}
%XXX: The four-point distribution that provides $P(\beta_k=r_k\mid g)$ is given in Appendix (???).

In the case that there is a sequencing error, we similarly see:
\begin{equation*}
    P(r_k,\, se \mid g) = P(r_k\mid se,\, g)e_k
\end{equation*}
Furthermore:
\begin{equation*}
    P(r_k\mid se,\, g) = P(\beta_k\neq r_k \mid se,\,g) P(r_k\mid \beta_k\neq r_k,\,se,\,g)
\end{equation*}
We expect an error in sequencing the intermediate allele could produce any of the other three alleles with equal probability.
Equivalently, we find $P(r_k\mid \beta_k \neq r_k,\,se,g)=1/3$.

Since the amplification of $\beta$ from the true underlying genotype is unaffected by sequencing:
\begin{equation*}
    P(\beta_k\neq r_k \mid se,\,g)=P(\beta_k\neq r_k\mid g)=1-P(\beta_k=r_k\mid g)
\end{equation*}
And therefore:
\begin{equation*}
    P(r_k, se \mid g) = e_k \frac{1}{3}\left[1-P(\beta_k=r_k\mid g)\right]
\end{equation*}
Finally the likelihood $P(d_{ij}\mid g)$ for cell at a locus for a homozygous genotype is:
\begin{equation}
     P(d_{ij}\mid g) = \prod_{k=1}^l \left[ (1-e_k)P(\beta_k=r_k\mid g) + e_k \frac{1}{3} (1-P(\beta_k=r_k\mid g)) \right]
\end{equation}
This equation is used for the homozygous cases $g\in\{0,2\}$.
This is somewhat more simple than the heterozygous case because an allelic dropout at a homozygous locus would not have a detectable effect on the read values and is therefore not considered.

\subsubsection*{Heterozygous genotypes and allelic dropout}
For the case where the underlying genotype is heterozygous, $g=1$, we must account for the possibility of an allelic dropout (ADO) event~\cite{monovar,sciphi}.
Allelic dropout is when from two homologous chromosomes in a cell only one is amplified, and the other is lost.
Since each chromosome is a single molecule of DNA in the cell, this is a very frequent event in single cell DNA sequencing.

For the heterozygous case we therefore first marginalize the genotype likelihood on whether or not such an ADO event has occurred:
\begin{equation*}
    P(d_{ij}\mid g=1) = P(d_{ij}, \text{ADO} \mid g=1)+ P(d_{ij}, \neg \text{ADO}\mid g=1)
\end{equation*}
Letting $P_{ADO}$ be the prior probability that any given locus in a cell has been affected by an allelic dropout event, this expands to:
\begin{equation*}
    P(d_{ij}\mid g=1) = P_{ADO}P(d_{ij}\mid \text{ADO},\, g=1) + (1-P_{ADO})P(d_{ij} \mid \neg \text{ADO},\, g=1)
\end{equation*}
$P_{ADO}$ is another initial parameter of SCarborSNV, and like in Monovar, is set by default to 0.2.
This value is set quite high to reflect how common these events are expected to be; if $P_{ADO}=0.2$ we expect that about four or more of the twenty-two homologous pairs of autosomal chromosomes in each cell will be affected by such a dropout.

If an ADO event affects a heterozygous locus, only one allele will remain after the amplification process and hence the likelihood $P(d_{ij}\mid \text{ADO},\, g=1)$ will resemble the homozygous case.
We assume allelic dropout can affect either  allele with equal probability and hence:
\begin{equation*}
    P(d_{ij}\mid \text{ADO},\, g=1) =\frac{1}{2}P(d_{ij}\mid g=0) + \frac{1}{2}P(d_{ij}\mid g=2)
\end{equation*}
For the case without allelic dropout, the form of the likelihood is identical to the homozygous case:
\begin{equation*}
      P(d_{ij}\mid \neg\text{ADO},\,g=1) = \prod_{k=1}^l \left[ (1-e_k)P(\beta_k=r_k\mid g=1) + e_k \frac{1}{3} (1-P(\beta_k=r_k\mid g=1)) \right]
\end{equation*}
With this final equation we can calculate the genotype likelihoods $P(d_{ij}\,\mid\,g)$ for any $g\in[0,2]$ for every cell at every locus.
These values can be then be used in a dynamic programming algorithm to calculate the likelihood of the total alternate allele count across each locus: $\Lambda(\sigma) = P(D_{i}\,\mid\,\sigma)$.
\end{document}
