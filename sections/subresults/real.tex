\documentclass[../../main.tex]{subfiles}

\begin{document}
While simulated data are very useful since a ground truth can be used to determine accuracy, it is important to see also how well SCarborSNV works on real data.
This however requires a significant amount of preprocessing to get the real data from a sequencing machine to be ready for SCarborSNV and its similar tools to use.
Furthermore, real datasets have significantly longer genomes than the simulated examples above.
Since Sci$\Phi$ in particular takes so long to run, for practical reasons we only compare SCarborSNV on a portion of the genome by truncating the input pileup file.
While this disadvantages both phylogeny aware algorithms by reducing the available kinship information, it should disadvantage them equally.
It should be noted, however, that this restriction may make Monovar perform comparatively better than if this test were to be done on the full genome.

\subsubsection*{Data and preprocessing}
We here use the same dataset used by \cite{sciphi} (accession: SRA053195), which includes DNA sequencing data from 16 single tumor cells from a triple negative breast cancer patient.
The raw FASTQ files were downloaded from the NCBI Sequence Read Archive using the SRAToolkit's fastq-dump program.

Next we indexed the hg19 human genome with the BWA index command, and then used BWA-mem to perform the Burrows-Wheeler alignment, aligning the reads from each of the cell samples to the now indexed hg19~\cite{BWAMEM}.
We then added unique read-group identifiers to each aligned BAM file created by BWA using the Picard tool `AddOrReplaceReadGroups'.

\large{TODO}

\end{document}

