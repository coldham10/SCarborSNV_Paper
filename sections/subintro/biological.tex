\documentclass[../../main.tex]{subfiles}

\begin{document}
While individual organisms are mostly genetically homogeneous, mutagenic factors can cause the genotypes of cells within an organism to diverge from the germline.
These somatic mutations are often inconsequential, but when they cause a cell to gain ``an autonomous will to divide, this aberrant uncontrolled cell division [can create] masses of tissue (tumors) that invade organs and destroy normal tissue."~\cite{EmperorMaladies}. 
Such a collection of rapidly dividing cells will share all the genetic mutations of their ancestors, including both germline mutations and the set of somatic mutations that caused the expansion.
In the context of tumor evolution, these mutations common to all cells in a tumor are known as clonal, truncal or public mutations.
%TODO reference everywhere clonal or truncal mutations named.
The set of all cells that contain clonal mutations is known as the clone.
As the cells continue to divide rapidly, some cells in the clone will undergo further somatic mutations, known as subclonal mutations.
Any subset of tumor cells descended from such a further mutated cell (all containing the subclonal mutation) are referred to as a subclone~\cite{bigbang, 21breasts, metastatic}.

After the tumor has had sufficient time to develop further mutations, the genetic landscape of the cells within the tumor will be one of hierarchical subclones; the mutations within any given cell will correspond with its specific ancestry of subclonal mutations within the tumor.
At the time of biopsy, there are often many different subclones, each with a unique ancestry and thus a unique set of clonal and subclonal mutations.
This genetic variation between cells in the tumor is called intra-tumor heterogeneity, and is of great importance for understanding tumor development, metastasis and treatment~\cite{metastatic, 21breasts, greaves2012clonal, merlo2006cancer}.
%TODO cite individual therapies comment, general citation for ITH

Since the genetic mutations by themselves leave few or no clues as to when in the tumor's history they occurred, resolving this subclonal structure can be challenging.

%Neighbour joining, biological data, etc. ploidy changes, state of the art, infinite sites(SNV and SNP don't occur on same locus), prior values, dynamic programming, massive parallelism (avoid read collisions by having workers do lots of work between reading loci.), assumption of no polyploidy. Which parameters are reestimateable? all?
\end{document}

