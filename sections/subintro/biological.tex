\documentclass[../../main.tex]{subfiles}

\begin{document}
%TODO better intro!
While we frequently consider organisms to be genetically homogeneous, mutagenic factors can cause the genetics of cells within an organism to diverge from the germline genotype. These somatic mutations are often inconsequential, but when they cause a cell to gain ``an autonomous will to divide, this aberrant uncontrolled cell division [can create] masses of tissue (tumors) that invade organs and destroy normal tissue."~\ref{EmperorMaladies}. 

%Cancer is caused by somatic genetic mutations and as a tumour grows it develops further mutations. Some of these somatic mutations are clonal or truncal mutations, which have occured in a cell that is a common ancestor of all extant tumour cells and thus affect all the cells in the tumour. Other mutations are subclonal, only a portion of the extant tumour cells share a lineage with the cell in which the mutation occured and as such only this fraction contain\\

%A a tumour grows, some of its cells develop further somatic mutations such as single nucleotide variations (SNVs) and copy number changes. The cells descended from these further mutated cells therefore form genetically distinct subclonal populations which are subject to a Darwinian evolutionary process. At the time of tumour biopsy, then, the cells removed will represent a genetically heterogeneous sample from the leaves of a subclonal tumour phylogeny, as well as any healthy cells accidentally introduced. While bulk DNA sequencing has been coupled with\\

%Neighbour joining, biological data, etc. ploidy changes, state of the art, infinite sites(SNV and SNP don't occur on same locus), prior values, dynamic programming, massive parallelism (avoid read collisions by having workers do lots of work between reading loci.), assumption of no polyploidy. Which parameters are reestimateable? all?
\end{document}

