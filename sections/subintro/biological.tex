\documentclass[../../main.tex]{subfiles}

\begin{document}
%TODO better intro!
While individual organisms are mostly genetically homogeneous, mutagenic factors can cause the genotypes of cells within an organism to diverge from germline.
These somatic mutations are often inconsequential, but when they cause a cell to gain ``an autonomous will to divide, this aberrant uncontrolled cell division [can create] masses of tissue (tumors) that invade organs and destroy normal tissue."~\cite{EmperorMaladies}. 
Such a collection of rapidly dividing cells will share all the genetic mutations of their ancestors, including both germline mutations and the set of somatic mutations that caused the expansion.
In the context of tumor evolution, these mutations are known as clonal (or truncal) mutations as they are common to all cells in the clone: the set of all cells clonally descended from the original mutant cell.
%TODO reference everywhere clonal or truncal mutations named.
As the cells continue to divide rapidly, some cells will undergo further somatic mutations.
These are known as subclonal mutations; any subset of tumor cells descended from a further mutated cell (all containing the subclonal mutation) are referred to as a subclone.

After the tumor has had sufficient time to develop further mutations, the genetic landscape of the cells within the tumor will be one of heirarchical subclones; the mutations within any given cell will correspond with its specific ancestry of subclonal mutations within the tumor.
At the time of biopsy, there are often a large number of different sublones, each with a unique ancestry and thus a unique set of clonal and subclonal mutations.
This genetic variation between cells in the tumor is called intra-tumor heterogeneity, and is of great importance for understanding tumor development, metastasis and treatment~\cite{metastatic}.
%TODO cite individual therapies comment, general citation for ITH

%Cancer is caused by somatic genetic mutations and as a tumour grows it develops further mutations. Some of these somatic mutations are clonal or truncal mutations, which have occured in a cell that is a common ancestor of all extant tumour cells and thus affect all the cells in the tumour. Other mutations are subclonal, only a portion of the extant tumour cells share a lineage with the cell in which the mutation occured and as such only this fraction contain\\

%A a tumour grows, some of its cells develop further somatic mutations such as single nucleotide variations (SNVs) and copy number changes. The cells descended from these further mutated cells therefore form genetically distinct subclonal populations which are subject to a Darwinian evolutionary process. At the time of tumour biopsy, then, the cells removed will represent a genetically heterogeneous sample from the leaves of a subclonal tumour phylogeny, as well as any healthy cells accidentally introduced. While bulk DNA sequencing has been coupled with\\

%Neighbour joining, biological data, etc. ploidy changes, state of the art, infinite sites(SNV and SNP don't occur on same locus), prior values, dynamic programming, massive parallelism (avoid read collisions by having workers do lots of work between reading loci.), assumption of no polyploidy. Which parameters are reestimateable? all?
\end{document}

