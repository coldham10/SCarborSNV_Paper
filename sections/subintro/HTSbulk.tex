\documentclass[../../main.tex]{subfiles}
%High throughput sequencing and bulk approaches

\begin{document}
%TODO check facts on HTS
While other approaches have tried to read long sequences of DNA (or coDNA)(?) directly, the high throughput sequencing (HTS) or next generation sequencing (NGS) approach is to amplify the original sequence many times before breaking it down into smaller overlapping subsequences and reassembling the short reads therefrom in silico.
This shotgun approach of HTS has proved far faster and cheaper than other methods, as well as being highly accurate (depending on how many short reads cover a given site).
It is possible to reconstruct de novo sequences by piecing together the overlaps of the short reads, but for known organisms the reads can be quickly aligned to a reference genome, with some flexibility to account for mutations and variations.

The standard approach for HTS is to extract the DNA from many thousands or millions of cells, amplify the region to be sequenced a few times (e.g. by PCR) and finally align the short reads. This will hereafter be referred to as bulk sequencing, as the genetic material is extracted from a bulk tissue sample (as oposed to from individual cells).
Bulk sequencing is very common practice, requires no specialized technology to separate cells, and acheives a high level of calling accuracy with a high read depth (many reads mapped to each locus).
Because of this bulk HTS data has been used to reconstruct tumor phylogenies, for example by ClonEvol.
At each locus, the proportion of reads containing mutant reads is used as an ersatz for the number of mutant cells.
These values can be used to discover clusters of mutations with similar frequencies, and potential orderings or phylogenies of these mutations can be evaluated based on how likely it is each cluster descends from the others.

Bulk analyses have provided insight into ITH, however the results are difficult to verify and ????? smaller subclones may be missed. TODO TODO TODO.
To resolve ITH more directly, the genotype of individual cells must be examined.

\end{document}

