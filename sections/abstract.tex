\documentclass[../main.tex]{subfiles}

\begin{document}

\abstract{Ongoing somatic mutations during cancer development lead to genetically distinct subclonal populations of cells within a tumour, each with a distinct subset of acquired mutations. These subclonal populations genetically diverge as new mutations occur and are subject to Darwinian selection pressures. This leads to a complex intra-tumour heterogeneity, the subclonal architecture of which is important for understanding cancer evolution and developing individualised therapies. Some approaches have used bulk DNA sequencing coupled with advanced clustering techniques to attempt to tease out this structure. Recent advances in single-cell DNA sequencing (SCS), however, have allowed new approaches such as Monovar and SCI$\Phi$ to examine this heterogeneity directly, despite the inherent low quality of the SCS data. We here present a new probabilistic algorithm, SCarborSNV, which we expect will efficiently call point mutations using aligned SCS sequencing data from a sample of multiple cells. After calling candidate variant loci using a detailed prior, SCarborSNV uses a neighbour joining algorithm to reconstruct a phylogeny which is used to genotype individual cells. We will compare SCarborSNV with existing methods on simulated and real data and expect to show that SCarborSNV performs competitively in calling single cell genotypes.}

\newpage

\end{document}
