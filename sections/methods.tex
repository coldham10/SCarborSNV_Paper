\documentclass[../main.tex]{subfiles}

\begin{document}

%\subfile{sections/submethods/method_introduction}
TODO: Polish method introduction
\subsection{Preprocessing and parallelism}
\subfile{sections/submethods/preprocess}

\subsection{Cell genotype likelihoods}
\subfile{sections/submethods/cell_likelihoods}

\subsection{Mutated site priors}
\subfile{sections/submethods/mutated_priors}

\subsection{Welltype site priors}
\subfile{sections/submethods/well_priors}

\subsection{Variant candidate site calling}
\subfile{sections/submethods/site_calling}

\subsection{Building a cell phylogeny}
\subfile{sections/submethods/phylo1}

%TODO does binomial model account for large number of potential homozygotes?
%TODO make sure nothing penalizes read depth
%TODO Idea by Yufeng: use a more restrictive filter on sites to build the tree, then open up to more for sequencing

%\subsection{Parameter reestimation and second cell phylogeny}
%is this necessary? Overall mutation rate, $P(H_t\mid H),\,P(H),\,\lambda,\,P(SNP_T\mid SNP),\,\dots$\\
%To some extent can be derived from papers like 21 breasts and metastatic. Can be reestimated?\\
%do we just reestimate parameters or update priors on cell loci? updating priors could be fun, but maybe wait til completed to see if it brings additional benefit.\\
%E.g. for each site: likelihood clonal = $\prod_j^m 1-P(g=0)$. (?) If likelihood clonal > likelihood not clonal, add 1 to clonal count. count/n = new prior on clonal mutations. Can do similar ML/EM estimations for all params, get new priors, re estimate posteriors.... Do at end. Note with new priors do not have to redo DP, as memoized values do not change(?)
%TODO implement this if necessary and if time permits
%TODO to estimate params post sampling candidates, can just use expectation from posteriors. E.g. P(SNV)/site = average 1-P(sig=0). P(LOH) at site = 3*p(g=2) at site = 3*[1-\prod_j[p(g_j=0)+p(g_j=1)]

\subsection{Genotyping single cells}
\subfile{sections/submethods/inference}

\subsection{Additional computational methods}
Stirlings approximation for $T(a,b)$. Pre computation /memoization of priors where possible. Log space.


\end{document}
